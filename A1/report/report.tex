\documentclass[12pt]{article}

\usepackage{graphicx}
\usepackage{paralist}
\usepackage{listings}
\usepackage{booktabs}
\usepackage{hyperref}

\oddsidemargin 0mm
\evensidemargin 0mm
\textwidth 160mm
\textheight 200mm

\pagestyle {plain}
\pagenumbering{arabic}

\newcounter{stepnum}

\title{Assignment 1 Solution}
\author{Cassidy Baldin, baldic1}
\date{\today}

\begin {document}

\maketitle

This report discusses testing of the \verb|ComplexT| and \verb|TriangleT|
classes written for Assignment 1. It also discusses testing of the partner's
version of the two classes. The design restrictions for the assignment
are critiqued and then various related discussion questions are answered.

\section{Assumptions and Exceptions} \label{AssumptAndExcept}

Some assumptions were made in the process of making these classes written for 
this assignment. These assumptions are made in the code where applicable, which 
can also be seen below as:
\begin{itemize}
        \item Input values passed to the \verb|ComplexT| constructor, were 
        assumed to be of type \verb|float|, and that both input parameters 
        would not be equal to 0 simultaneously (input will never be z = 0 + 0i),
        \item Return values of \verb|get_phi| were assumed to be in the range 
        of (-pi, pi],
        \item Input values for method \verb|equal| in class \verb|ComplexT| is 
        of type \verb|ComplexT|, and values are considered to be equal if both 
        the real and imaginary values of the argument are equal to the current 
        real and imaginary values respectively (within 9 decimal places),
        \item Return value of \verb|conj|, \verb|add|, \verb|sub|, \verb|mult|, 
        \verb|recip|, \verb|div| and \verb|sqrt| is a new ComplexT that is 
        equal to the result of their respective methods,
        \item Return value of \verb|sqrt| is only the positve part of the square 
        root of the current object, 
        \item Input values passed to the \verb|TriangleT| constructor, were 
        assumed to be positive, non-zero integer values (to relate to the real 
        world application of triangle dimensions),
        \item Input values for method \verb|equal| in class \verb|TriangleT| is 
        of type \verb|TriangleT|, and values are considered to be equal all side 
        lengths are equal )within 9 decimal places),
        \item Input value for method \verb|area| is a valid triangle so that the 
        area can actually be calculated, 
        \item Priority labels for method \verb|tri_type| (when a given triangle 
        can have more than one label) is in the order right, equilateral, 
        isosceles, then scalene
        \item Input value for method \verb|tri_type| is a valid triangle (so 
        that it can be classified accordingly, as there is no option for it 
        to be of type \verb|notvalid|).
\end{itemize}

\section{Test Cases and Rationale} \label{Testing}

Tests were written such that each method that was implemented into the design
had an appropriate amount of test cases that I felt covered the edge/boundary 
cases for each method respectively. 

\section{Results of Testing Partner's Code}


\section{Critique of Given Design Specification}


\section{Answers to Questions}

\begin{enumerate}[(a)]

\item 
\item ...

\end{enumerate}

\newpage

\lstset{language=Python, basicstyle=\tiny, breaklines=true, showspaces=false,
  showstringspaces=false, breakatwhitespace=true}
%\lstset{language=C,linewidth=.94\textwidth,xleftmargin=1.1cm}

\def\thesection{\Alph{section}}

\section{Code for complex\_adt.py}

\noindent \lstinputlisting{../src/complex_adt.py}

\newpage

\section{Code for triangle\_adt.py}

\noindent \lstinputlisting{../src/triangle_adt.py}

\newpage

\section{Code for test\_driver.py}

\noindent \lstinputlisting{../src/test_driver.py}

\newpage

\section{Code for Partner's complex\_adt.py}

\noindent \lstinputlisting{../partner/complex_adt.py}

\section{Code for Partner's triangle\_adt.py}

\noindent \lstinputlisting{../partner/triangle_adt.py}

\end {document}